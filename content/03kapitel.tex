%!TEX root = ../dokumentation.tex
% -------------------------------
\chapter{Design} % ~4–5 pages, ~1600–2000 words
% -------------------------------
\label{chap:design}


\section{Requirements}
% Aus Kapitel 1 breiter ausgeführt, ASTRA_sim bezogen, usability etc.

\section{Requirements Analysis}
% zb viele excel files = Graphs idk -> konkret darauf eingehen was gebraucht wird

\section{Technology Selection}
% grommet = branding
% flask API = einfacher astra sim zugang
% react = wegen grommet
% TS = wegen data consistency ( evtl braucht astra sim des halt wegen viele inputs und so)
% python = vorliegende scripte, simple API
% Docker = Container vs keine = easy ASTRA-sim deployment (installationen dependencies etc), Separation of concerns
% Figma = usability pretesting, early feedback & user tests

\begin{figure}[H]
    \centering
    \includegraphics[width=0.7\textwidth]{figures/methodology_flowchart.png}
    \caption{Methodology flowchart}
\end{figure}

\section{UI/UX Design Process}
% Wireframes, heuristics ??? TODO check this comment
% Immer Bezug auf Kapitel 2.3 nehmen! Wegen wissenschaft und so
% 1. Overview (Branding & Wizard)
% 2. ASTRA-sim Inputs (analyse (prio, thema weil falsch), beispiel & default werte (tablellen), UI entscheidungen Input type)
% 3. ASTRA-sim Outputs (analyse, visualisierungs  möglichkeiten, evtl data science block für weitere extensions denn das würde python legimatisieren)
% 4. Extended Features: Import & export, History
% 5. Umsetzung: Wireframe first idea what was bad, next idea, what was bad now idea what is bad oder so

\begin{figure}[H]
    \centering
    \includegraphics[width=0.8\textwidth]{figures/cli_vs_ui_workflow.png}
    \caption{Concept diagram: CLI workflow vs UI workflow}
\end{figure}

