%!TEX root = ../dokumentation.tex

% -------------------------------
\chapter{Introduction} % ~4–5 pages, ~1600–2000 words
% -------------------------------

% Motivation
% Problem Statement
% Objectives
% Contributions
% Structure of the Thesis

\section{Motivation}
% 1. ML scaled higher = Distributed ML
Machine Learning is used in many aspects of daily life, business and research. To fit demands of high scalability, distribution of the training is used. 
% 2. Distributed ML dependent on many factors, expensive tryouts = Simulation
\ac{DML} is dependable of configurations such as parallelization strategy, topologies and communication. Important is to find configurations to minimize used training time and optimize computation and communication distribution.
Finding such configurations is difficult, as it depends on an expensive hardware base and a huge variety of available system choices. Optimizing choices uses many resources, such as money, time and power. To reduce these costs, simulation can be used. One simulator for research of \ac{DML} is \ac{ASTRA-Sim}. 
% 3. HPEs customers have also ML models, to sell more infrastructure to the, HPE can provide them with according expected DML times / cycles = Usage of simulation tools such as ASTRA-Sim
When companies want to train their own \ac{ML} model, they face similar challenges. Here \ac{HPE} could provide solutions such as the necessary hardware infrastructure, making those companies their customers. Additionally, they should provide associated system choices to use the system efficiently. To gain the companies as their customers, they should provide realistic insights in how long the training would take. To find and help evaluate such information, a simulator like \ac{ASTRA-Sim} can be used. 
% 4. ASTRA-sims use case is very different, making it not accessible for HPE presales = UI
As it is a research tool and not designed for a sales use case like this, its usage is challenging. It's multiple versions and parameters, that need prior training and expertise, make usage for a non-expert user group like \ac{HPE} customers difficult.

\section{Problem Statement}
% Tagret user group: Presales
% ASTRA-sim:  many versions, many inputs, many outputs, no conventions, much prior knowldge, training, outputs would have to be made pretty for customers, research tool has different focus so maybe extend functionality, incomplete documentation

\section{Objectives}
% User interface that is:
% 1. Verständlich = ohne viel Hintergrundwissen
% 2. Einfach nutzbar = Usability regeln befolgen
% 3. Vertrauenswürdig = HPE Branding etc (schau in Präsi? wenn das nicht hilft hör Audio)

% Long run: Fusion of all ASTRA-sim Versions to one User Interface

% \begin{figure}[H]
%     \centering
%     \includegraphics[width=0.5\textwidth]{figures/persona_sketch.png}
%     \caption{Illustration of target user persona}
% \end{figure}

\section{Structure}
% Brief overview of each chapter


